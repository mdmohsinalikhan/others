\title{Linear regression of the outputs of a device and optimizing the linear model using linear programming}
\author{
        Dristy Parveg\\
        Mohsin Khan\\
\underline{Finland}
}
\date{\today}

\documentclass[12pt]{article}
\usepackage[]{xcolor}
\usepackage{amsfonts}
\usepackage[]{amsmath}
\usepackage{graphicx}
\usepackage{amssymb}



\begin{document}
\maketitle
\section{Linear Modeling and optimization:}
It has been observed that the device has two positive outputs $y_1,y_2 \in \mathbb{R}$. We have learnt that $y_1$ is a function of $5$ positive variables $x_1,x_2,x_3,x_4,x_5 \in \mathbb{R}$ and we name this function $F$. We also have learnt that $y_2$ is a function of $3$ variables $x_1,x_2,x_3$ and we name this function $G$. However, we don't exactly know the definitions of $F$ and $G$. Nonetheless we can sample these two functions at different input points through experiments as discussed in the previous sections. We have results of $18$ experiments. We are interested to optimize the functions $F$ and $G$. We are interested to optimize the functions in few alternative ways. The candidates are as follows:
\begin{enumerate}
\item $Maximize \text{ }F - G$: Finding values $o_1,o_2,o_3,o_4,o_5$ so that $F(o_1,o_2,o_3,o_4,o_5) - G(o_1,o_2,o_3)$ becomes as small as possible.
\item $Minimize \text{ } \vert\frac{F}{G}-d\vert$: Finding the values $o_1,o_2,o_3,o_4,o_5$ so that $\frac{F(o_1,o_2,o_3,o_4,o_5)}{G(o_1,o_2,o_3)}$ becomes as close as possible to a certain constant $d$
\item $Minimize \text{ } \vert F-d_1 \vert + \vert G-d_2 \vert$: Finding the values $o_1,o_2,o_3,o_4,o_5$ so that $F(o_1,o_2,o_3,o_4,o_5)$ and $G(o_1,o_2,o_3)$  goes as close as possible to constant values $d_1$ and $d_2$ respectively.
\end{enumerate}
\paragraph{}
Let $f_i:\mathbb{R} \rightarrow \mathbb{R}$ be a function that takes the variable $x$ as input while the other input variables $x_{j} \text{ for } j \in \left\lbrace1,...5\right\rbrace\setminus{i}$ of $F$ are kept at constant values i.e., $f_i(x\vert a_1,...,a_{i-1},x,a_{i+1},...,a_{5}) = F(a_1,...,a_{i-1},x,a_{i+1},...,a_{5})$. We define the function $g_i:\mathbb{R} \rightarrow \mathbb{R}$ in the same way so that $g_i(x\vert a_1,...,a_{i-1},x,a_{i+1},...,a_{5}) = G(a_1,...,a_{i-1},x,a_{i+1},...,a_{5})$. From the result of the experiments, presented in previous section, it is evident that $f_i$ and  $g_i$ are almost, if not exactly, linear functions.
\paragraph{}
We are interested to find models $f:\mathbb{R}^5 \rightarrow \mathbb{R}$, $g:\mathbb{R}^3 \rightarrow \mathbb{R}$ where $f(x_1,x_2,x_3,x_4,x_5) = G(x_1,x_2,x_3,x_4,x_5)$ and $g(x_1,x_2,x_3) = G(x_1,x_2,x_3)$. We make an educated guess that both $F$ and $G$ are linear. We do not provide an analytical argument to justify the linearity of $F$ and $G$. Instead, we use linear regression on the $18$ experimental results and obtain the linear models
\begin{eqnarray*}
f(x_1, x_2, x_3, x_4, x_5) &=& -7.2667 \cdot x_1 + 0.3724 \cdot x_2 + 29.1759 \cdot x_3 - 3.2784 \cdot x_4\\
& & + 1.8231 \cdot x_5 + 78.2264 \\
g(x_1, x_2, x_3) &=& -5.477 \cdot x_1 + 4.640 \cdot x_2 + 4.290 \cdot x_3 + 24.343
\end{eqnarray*}
For an input $(x_1,x_2,x_3,x_4,x_5)$, the error percentages of the models $f$ and $g$ are $\frac{|f(x_1,x_2,x_3,x_4,x_5)-F(x_1,x_2,x_3,x_4,x_5)|}{F(x_1,x_2,x_3,x_4,x_5)} \times 100$ and $\frac{|g(x_1,x_2,x_3)-G(x_1,x_2,x_3)|}{G(x_1,x_2,x_3)}\times 100$ respectively. We compute the mean, variance, minimum and maximum error percentages of the models $f$ and $g$ over all the $18$ inputs and present in the below table

\begin{center}
    \begin{tabular}{ | l | l | l | l | l |}
    \hline 
		& &\multicolumn{3}{ |c| }{Error percentage} \\
    \hline
    Model & Mean & Standard Deviation & Min  & Max\\ \hline
    $f$ & 2.209766 & 2.480924 & 0.5398369 & 9.525519 \\ \hline
    $g$ & 1.002286 & 1.329765 & 0.008706377 & 5.115356 \\ \hline
    \end{tabular}
\end{center}

\paragraph{}
We reckon, these error percentages indicate that the underlying functions $F$ and $G$ are fairly linear and the obtained models $f$ and $g$ are fairly good linear approximations of $F$ and $G$. Consequently, we optimize $F$ and $G$ by optimzing $f$ and $g$ respectively. As $F$ and $G$ are not not exactly linear and our obtained models $f$ and $g$ are linear approximations, the optimal values obtained from the models $f$ and $g$ will have some errors comparing to the optimal values of $F$ and $G$. We expect these errors to be similar to the errors of the models.
\paragraph{}
As both $f$ and $g$ are linear and all the inputs of both of the functions are positive, the optimization problems are most likely to be representable as a linear programming problem. The linear programming problem would involve $5$ variables. A linear programming problem with $5$ variables is a small instance of a linear programming problem. There exist efficient algorithms, namely, simplex, ellipsoid, etc. for solving linear programming problems. In practice, a linear programming problem can easily be solved by any standard linear programming solver like lp\_solve. 
\paragraph{}
Now we present the optimization problems discussed above as s linear programming problems in the following subsections:
\subsection{$Maximize \text{ }f - g$:}
We need to find values $o_1,o_2,o_3,o_4,o_5$ so that $f(o_1,o_2,o_3,o_4,o_5) - g(o_1,o_2,o_3)$ becomes as small as possible, i.e., we need to maximize
\begin{eqnarray*}
& & f(x_1,x_2,x_3,x_4,x_5) - g(x_1,x_2,x_3)\\
&=& (-7.2667 \cdot x_1 + 0.3724 \cdot x_2 + 29.1759 \cdot x_3 - 3.2784 \cdot x_4\\
& & + 1.8231 \cdot x_5 + 78.2264)\\
& & - (-5.477 \cdot x_1 + 4.640 \cdot x_2 + 4.290 \cdot x_3 + 24.343)\\
&=& -1.7897 \cdot x_1 - 3.9176 \cdot x_2 + 24.8859 \cdot x_3\\
& & - 3.2784 \cdot x_4 + 1.8231 \cdot x_5 + 53.8834
\end{eqnarray*}

 Consequently the linear programming problem looks as follows
\begin{eqnarray*}
\text{max:} & & -1.7897 \cdot x_1 - 3.9176 \cdot x_2 + 24.8859 \cdot x_3 - 3.2784 \cdot x_4 + 1.8231 \cdot x_5 + 53.8834;\\
\\
& & x_1 <= 5;\\
& & x_2 <= 5;\\
& & x_3 <= 3;\\
& & x_4 <= 15;\\
& & x_5 <= 21;\\
& & x_1 >= 1;\\
& & x_2 >= 1;\\
& & x_3 >= 1;\\
& & x_4 >= 2;\\
& & x_5 >= 5;\\
\end{eqnarray*}
The above linear programming problem can be solved by saving the above problem in a file named \textit{model.lp} and run the command \textit{lp\_solve model.lp}. We get the maximum value of the objective function to be: $154.562$ at the input $x_1 = 1,x_2 = 1,x_3 = 3, x_4 = 2, x_5 = 21$. Feeding this input to $f$ and $g$ we find
\begin{eqnarray*}
f(1,1,3,2,21) &=& -7.2667 + 0.3724 + 29.1759 \times 3 - 3.2784 \times 2\\
& & + 1.8231 \cdot 21 + 78.2264\\
&=& 190.5881\\
g(1,1,3) &=& -5.477 + 4.640 + 4.290 * 3 + 24.343 = 36.376
\end{eqnarray*}

We can find the value of $F(1,1,3,2,21)$ and $G(1,1,3)$ by running an experiment to see how good the optimization is.


\subsection{$Minimize \text{ } \vert\frac{f}{g}-d\vert$:} We need to find the values $o_1,o_2,o_3,o_4,o_5$ so that $\frac{f(o_1,o_2,o_3,o_4,o_5)}{g(o_1,o_2,o_3)}$ becomes as close as possible to a certain constant $d$. Let us set $d=6$. Then we need to minimize $\frac{f(x_1,x_2,x_3,x_4,x_5)}{g(o_1,o_2,o_3)}-6$. In other words we need to minimize 
\begin{eqnarray*}
& & \vert f(x_1,x_2,x_3,x_4,x_5) - 6 g(x_1,x_2,x_3) \vert\\
&=& \vert(-7.2667 \cdot x_1 + 0.3724 \cdot x_2 + 29.1759 \cdot x_3 - 3.2784 \cdot x_4\\
& & + 1.8231 \cdot x_5 + 78.2264)\\
& & - 6(-5.477 \cdot x_1 + 4.640 \cdot x_2 + 4.290 \cdot x_3 + 24.343)\vert \\
&=& \vert 25.5953 \cdot x_1 - 27.4675 \cdot x_2 + 3.4358 \cdot x_3 - 3.2784 \cdot x_4\\
& & + 1.8231 \cdot x_5 - 67.8316 \vert
\end{eqnarray*}
 As a result the linear programming problem looks as follows:
\textcolor{red}{the linear program is written in model1.lp file} 
the solution is:
$x_1 = 3.4891, x_2 = 1, x_3 = 1, x_4 = 2, x_5 = 5 $. The optimized value of the linear program is $0$. And
$f(x_1 = 3.4891, x_2 = 1, x_3 = 1, x_4 = 2, x_5 = 5) = 84.97916$\\
$g(x_1 = 3.4891, x_2 = 1, x_3 = 1) = 14.1632$

We can find the value of $F(3.4891,1,1,2,5)$ and $G(3.4891,1,1)$ by running an experiment to see how good the optimization is.

\subsection{$Minimize \text{ } \vert f-d_1 \vert + \vert g-d_2 \vert$:} Finding the values $o_1,o_2,o_3,o_4,o_5$ so that $f(o_1,o_2,o_3,o_4,o_5)$ and $g(o_1,o_2,o_3)$  goes as close as possible to constant values $d_1$ and $d_2$ respectively. Let us set $d_1 = 120$ and $d_2 = 24$. We need to minimize
\begin{eqnarray*}
& &\vert f(x_1,x_2,x_3,x_4,x_5)-120 \vert + \vert g(x_1,x_2,x_3)-24 \vert \\ 
&=& \vert -7.2667 \cdot x_1 + 0.3724 \cdot x_2 + 29.1759 \cdot x_3 - 3.2784 \cdot x_4 + 1.8231 \cdot x_5 + 78.2264 - 120 \vert \\
& & + \vert -5.477 \cdot x_1 + 4.640 \cdot x_2 + 4.290 \cdot x_3 + 24.343 -24 \vert \\
& = &  \vert -7.2667 \cdot x_1 + 0.3724 \cdot x_2 + 29.1759 \cdot x_3 - 3.2784 \cdot x_4\\
& & + 1.8231 \cdot x_5 - 41.776 \vert + \vert -5.477 \cdot x_1 + 4.640 \cdot x_2 + 4.290 \cdot x_3 + 0.343 \vert
\end{eqnarray*}
So, the linear program looks as follows: \textcolor{red}{the linear program is in the file model2.lp}¸
The optimized value of the linear program is $0$. The value of the variables are $X = 0, Y = 0, x_1 = 2.42593, x_2 = 1, x_3 = 1.93562, x_4 = 2, x_5 = 5$. And \\
$f(x_1 = 2.42593, x_2 = 1, x_3 = 1.93562, x_4 = 2, x_5 = 5) = 120.0025$\\
$g(x_1 = 2.42593, x_2 = 1, x_3 = 1.93562) = 23.99999$

We can find the value of $F(2.42593, 1, 1.93562, 2, 5)$ and $G(2.42593, 1, 1.93562)$ by running an experiment to see how good the optimization is.
\end{document}


